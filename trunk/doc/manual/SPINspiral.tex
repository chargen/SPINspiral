\documentclass[10pt]{article}
\usepackage[english]{babel}
\usepackage[letterpaper, top=2.0cm, bottom=2.0cm, left=2.0cm, right=2.0cm]{geometry}
%\usepackage[square,sort&compress]{natbib}

\title{Documentation \textsc{SPINspiral}:\\
  Parameter estimation on gravitational-wave signals\\
  from spinning binary inspirals}
\author{Marc van der Sluys, Vivien Raymond, Ilya Mandel \\
  Northwestern University \\
  \small\texttt{\href{http://www.astro.northwestern.edu/~sluys/index.php?title=SPINspiral}{http://www.astro.northwestern.edu/$\sim$sluys/index.php?title=SPINspiral}}}


\usepackage[pdftex=true]{hyperref}  
\hypersetup{
  pdftitle = Documentation SPINspiral: Parameter estimation on gravitational-wave signals from spinning binary inspirals,
  pdfauthor = Marc van der Sluys,
  pdfsubject = Documentation MCMC code SPINspiral,
  pdfkeywords = SPINspiral MCMC Documentation,
  colorlinks = true,
  linkcolor = blue,
  anchorcolor = blue,
  citecolor = blue,
  filecolor = blue,
  urlcolor = blue,
  pdfpagemode=UseThumbs,
}
 
\begin{document}

\maketitle




\begin{abstract}
  \textsc{SPINspiral} was developed at Northwestern University to analyse gravitational-wave signals from
  stellar-mass binary inspirals, as can be detected by ground-based interferometers like LIGO and Virgo.  
  The code performs parameter estimation on such signals, using Markov-chain Monte-Carlo (MCMC) techniques.
  This analysis includes the spins of the binary components.  \textsc{SPINspiral} was adapted from an MCMC 
  code for non-spinning binary inspirals written by Christian R\"over~\cite{RoeverThesis2007}.
\end{abstract}

\tableofcontents

\pagebreak


\section{Getting started}

\subsection{Quick start}

\begin{enumerate}
\item You will need to install the LIGO Analysis Library (LAL)~\cite{lal} if you want to use the LAL waveforms.
\item Check out a working copy from the \texttt{SVN} and cd to the directory \texttt{trunk}, or go to the
  directory \texttt{trunk} and update your working copy (\texttt{svn up}).
\item Check whether a proper \texttt{Makefile} is in your trunk.  There's an example file in \texttt{doc/}, 
  make sure it is configured for your compiler and type \texttt{make clean all} for the first compilation
  and \texttt{make all} for subsequent ones.  The executable files \texttt{mcmc\_nolal} and \texttt{mcmc\_lal} should be produced.
\item Make sure you have the configuration files \texttt{SPINspiral.input*} in your trunk.  There are example
  files in \texttt{doc/input\_*}.  Set \texttt{nIter} in \texttt{SPINspiral.input.mcmc} to 10 or 100 for a first test run.
  Example input files are available for both \texttt{mcmc\_nolal} and \texttt{mcmc\_lal}.
\item Type \texttt{./mcmc\_nolal} or \texttt{./mcmc\_lal} and \textsc{SPINspiral} should run with the parameters specified
  in the input files.
\end{enumerate}



\subsection{SVN}

\textsc{SPINspiral} is currently maintained at Northwestern in a version-control system called \texttt{Subversion} or \texttt{SVN}.  The
repository can be found at \texttt{https://www.astro.northwestern.edu/svn/sluys/mcmc\_code/}.
A quick \texttt{SVN} HOWTO with links for more info can be found at \\
\texttt{http://www.astro.northwestern.edu/$\sim$sluys/public/svn-howto.html}.  
The first time you want
to retrieve the source code from the server, you need to check it out with: \\ 
\texttt{svn checkout --username <user> https://astro.astro.northwestern.edu/svn/sluys/mcmc\_code/},
which will prompt for a password.  It will create a directory \texttt{mcmc\_code/} in your current 
directory, and the default working directory is \texttt{mcmc\_code/trunk/}.




\section{Source code}

\subsection{Source files}

\textsc{SPINspiral} currently has ten source files (\texttt{mcmc\_*.c}) and is written in plain C.
An example Makefile that should compile your code is included in the \texttt{SVN}, in the directory 
\texttt{doc/}.

\textsc{SPINspiral} is split into several routines and functions that give it a modular character.
The routines are grouped into six source files:

\begin{description}
\item[SPINspiral\_main.c] Main routine. The idea is to keep this routine as small as possible and use it to call other routines.
\item[SPINspiral\_parameters.c] Contains routines that deal with reading input files and choosing and setting (starting or injection) parameters.
\item[SPINspiral\_data.c] Contains routines for data and noise reading and handling.
\item[SPINspiral\_templates.c] Contains routines to generate templates or obtain them from LAL.
\item[SPINspiral\_signal.c] Contains routines to compute things like overlaps and likelihoods.
\item[SPINspiral\_mcmc.c] Contains the MCMC core.
\item[SPINspiral\_lal.c] Contains interfaces to LAL routines.
\item[SPINspiral\_nolal.c] Contains dummy interfaces to LAL routines, used when not linking against LAL.
\item[SPINspiral\_routines.c] Contains more general supporting functions.
\item[SPINspiral\_3rdparty.c] Contains third-party routines.
\end{description}





\pagebreak
\section{Command-line options}

Most of \textsc{SPINspiral}'s options ($\sim 250-300$) are specified in input files (see Section~\ref{sec:input_files}).
However, a small --- but possibly growing --- number of options can be passed to the program 
using command-line options:

\begin{description}
  \item[\texttt{-i <main input file name>}] override the name of the main input file (Section~\ref{sec:SPINspiral.input})
  \item[\texttt{--injXMLfilename <file name>}] specify an XML file to read injection data from
  \item[\texttt{--injXMLnr <0-\ldots>}] specify the injection number in the injection XML file
\end{description}
  
  















\pagebreak
\section{Input files}
\label{sec:input_files}

\textsc{SPINspiral} has six different ASCII input files.  The main input file is assumed to be called
\texttt{SPINspiral.input}, but can be specified differently using the command-line option \texttt{-i 
  <main input file name>}.  The names of the other five input files are specified in the main
input file.  Example input files can be found in the \texttt{doc/input\_*} directories in the svn
trunk.  The directory \texttt{doc/input\_all} contains the four waveform-independent files, the
other \texttt{doc/input\_*} directories contain the two files that specify the injection and MCMC
templates.


\subsection{SPINspiral.input}
\label{sec:SPINspiral.input}

The file \texttt{SPINspiral.input} is the main input file and specifies the main mode of operation 
and the names of the other input files.

\subsubsection{Operation and output}
\begin{description}
\item[doSNR] Calculate the SNR: 0-no, 1-yes. Default: 1.
\item[doMCMC] Do MCMC: 0-no, 1-yes.  Default: 1.
\item[doMatch] Calculate matches: 0-no, 1-yes.  Default: 0.
\item[intscrout]  Print initialisation output to screen: 0-no, 1-yes.  Default: 0.
\item[writeSignalWrite] signal, noise, PSDs to file: 0-no, 1-yes.  Default: 0.
\item[printMuch]  Print long stretches of output (1) or not (0).  Default: 0.
\end{description}


\subsubsection{Secondary input files}
\begin{description}
\item[mcmcFilename] Name of the MCMC input file.  Default: SPINspiral.input.mcmc.
\item[dataFilename] Name of the data/noise input file.  Default: SPINspiral.input.data.
\item[injectionFilename] Name of the software injection input file.  Default: SPINspiral.input.injection.
\item[parameterFilename] Name of the MCMC parameter input file.  Default: SPINspiral.input.parameters.
\item[systemFilename] Name of the file with system-dependent parameters.  Default: SPINspiral.input.system.
\end{description}








%\subsection{}
%\begin{description}
%\item[]
%\end{description}


\pagebreak
\subsection{SPINspiral.input.mcmc}                                                                                                                                               
The MCMC input file                                                                                                                                                 
                                                                                                                                                                    

\subsubsection{Basic settings}
\begin{description}
\item[nIter] Total number of iterations to be computed, \textit{e.g.} $10^7$.
\item[thinOutput] Number of iterations to be skipped between stored steps (100 for 1d).
\item[thinScreenOutput] Number of iterations between screen outputs in the MCMC (1000 for 1d).
\item[MCMCseed] Random number seed to start the MCMC: 0-let system clock determine seed, $>0$: use the specified seed.  Default: 0.
\item[adaptiveMCMC] Use adaptation: 0-no, 1-yes.  Default: 1.
\item[acceptRateTarget] Target acceptance rate for MCMC (0.0-1.0).  We used 0.25 for a long time.
\item[minlogL] Minimum value for the log Likelihood to accept a jump. We used 0 for a long time, this number shouldn't be positive! Try -1.e3.
\item[blockFrac] Fraction of uncorrelated updates that is updated as a block of all parameters ($\leq 0.0$: none, $\geq 1.0$: all).  Default: 0.1.
\end{description}


\subsubsection{Correlated update proposals}
\begin{description}
\item[correlatedUpdates] Do correlated update proposals: 0-no, 1-yes but update the matrix only once, 2-yes and update the matrix every \texttt{ncorr} iterations.  Default: 2.
\item[corrFrac] Fraction of update proposals that is correlated (0.0-1.0, ~0.7 seems OK). corrupd must be 2. Should this replace corrupd?  Default: 0.7.
\item[nCorr] Number of iterations for which the covariance matrix is calculated.  Default: $10^3-10^4$.
\item[matAccFr] Fraction of elements on the diagonal that must `improve' in order to accept a new covariance matrix. ???~0.6-0.8 for unimodal, 0.0-0.2 for multimodal???  Default: 0.5.
\item[prMatrixInfo] Print information to screen on proposed matrix updates: 0-none, 1-some (default), 2-add the old and new matrices.  Default: 1.
\end{description}

\subsubsection{Annealing}
\begin{description}
\item[annealTemp0] Starting temperature of the chain, \textit{e.g.} 100.0. Set 1.0 for no temperature effect.  Default: 1.0.
\item[annealNburn] Number of iterations for the burn-in phase (1e4) at this number, the temperature drops to 1.0.  Default: $10^5$.
\item[annealNburn0] Number of iterations during which temp=temp0 (e.g. 0.1*annealNburn, should be lower than ~0.9*annealNburn).  Default: $10^5$.
\end{description}

\subsubsection{Parallel tempering}
\begin{description}
\item[parallelTempering] Use parallel tempering:  0-no,  1-auto, fixed $T$ ladder,  2-auto, sinusoid $T$ ladder,  3-manual, fixed $T$ ladder,  4-manual, sinusoid $T$ ladder.  For a manual ladder, see near the bottom of the file.  Default: 2.
\item[nTemps] Number of steps in the temperature ladder for parallel tempering, typically 5-10.  Default: 5.
\item[maxTemp] Maximum temperature in automatic parallel-tempering ladder (equidistant in $\log(T)$), typically 20-100.  Default: 40.
\item[saveHotChains] Save hot ($T>1$) parallel-tempering chains: 0-no (just the $T=1$ chain), $>0$-yes; for every saved $T=1$ point, save every savehotchains-th hot point.  Default: 100.
\item[prParTempInfo] Print information to screen on the temperature chains: 0-none, 1-some ladder info (default), 2-add chain-swap matrix.  Default: 2.
\end{description}

\subsubsection{Manual temperature ladder for parallel tempering (tempLadder[])}
At least \texttt{nTemps} increasing temperature values, starting with 1.0, \textit{e.g.} \texttt{1.00     2.00     4.00     8.00     16.00}
    





\pagebreak
\subsection{SPINspiral.input.data}
\label{sec:input.data}
The data/noise input file


\subsubsection{General}
\begin{description}
\item[datasetName] Name of the data set used (for printing purposes), up to 80 characters, \textit{e.g.} NINJA data set
\end{description}

\subsubsection{Detector network}
\begin{description}
\item[networksize] Set the number of detectors that make up the network; read in networksize block of IFO data below (Currently 1--3)
\item[selectifos] Select the IFOs to use  1: H1, 2: L1, 3: V, \textit{e.g.} \texttt{1 2 3} 
\end{description}

\subsubsection{Data handling}
\begin{description}
\item[downsamplefactor] Downsample the sampling frequency of the detector (16-20kHz for the detectors, 4kHz for NINJA) by this factor.  Default (for detectors): 4.0. 10+1.4Mo needs $\sim 16\times$ a<0.1, 8x: $a\leq 0.8$, 4x: $a>0.8$. \textbf{Notice the difference of a factor of $\sim4$ in the original sampling rate between detector data and \textit{e.g.} the NINJA data files!}
\item[databeforetc] The stretch of data in seconds before presumed coalescence that is read in as part of the data segment, \textit{e.g.} 6.0.
\item[dataaftertc] The stretch of data in seconds after presumed coalescence that is read in as part of the data segment, \textit{e.g.} 1.0.
\item[lowfrequencycut] Templates and overlap integration start at this frequency, \textit{e.g.} 40.0.
\item[highfrequencycut] Overlap integration ends at this frequency, \textit{e.g.} 400.0.
\item[tukeywin] Parameter for Tukey-window used in \texttt{dataFT()} (non-flat fraction of window);  Use 0.15 for Virgo data.  Default: 0.15.
\end{description}

\subsubsection{Noise PSD estimation}
\begin{description}
\item[PSDsegmentNumber] Number of data segments used for the PSD estimation.  Default: 32, quick test runs: 8 or 4.
\item[PSDsegmentLength] Length of each data segment used for PSD estimation.  Default: 8.0 $\rightarrow$ 32x8.0s = 256s, quick test runs: 4.0 or 2.0 (8x2.0 = 16s; 4x1.0s = 4s).
\end{description}




\subsubsection{IFO i}
\label{sec:input.data.ifo}
Parameters for the location, orientation and data file for each detector used
(see also Sect.\,\ref{sec:frame}):


\begin{description}
\item[name] Detector name, \textit{e.g.} Hanford.
\item[lati] Latitude (degrees), \textit{e.g.} 46.45.
\item[longi] Longitude (degrees), \textit{e.g.} 119.41.
\item[rightarm] Orientation of the 'right' arm (degrees), \textit{e.g.} 36.80.
\item[leftarm] Orientation of the 'left' arm (degrees), \textit{e.g.} 126.80.
                                                                                                                                                                                                               
\item[ch$1$name] Name of the data channel in the Frame file, \textit{e.g.} \texttt{H1:STRAIN} (see Sect.\,\ref{sec:frame}).
\item[ch$1$filepath] Subdirectory of the path in \texttt{SPINspiral.input.system} where the data sits.  Use ``.'' for no subdirectory.  Default: ``.''.
\item[ch$1$fileprefix] Prefix of the Frame data file name, \textit{e.g.} \texttt{H-H1\_NINJA\_NOISE} (see Sect.\,\ref{sec:frame}).
\item[ch$1$filesuffix] Suffix of the Frame file name, \textit{e.g.} \texttt{-1024.gwf} (see Sect.\,\ref{sec:frame}).
\item[ch$1$filesize] `Size' (in seconds) of the data in each data Frame file, \textit{e.g.} 1024.
\item[ch$1$fileoffset] If the Frame file name ends in: -839366009-128.gwf (where 128 is the length of the data stream), \texttt{fileoffset} = \texttt{mod(839366009,128)}, \textit{e.g.} 743 (see Sect.\,\ref{sec:frame}).
\item[ch$1$doubleprecision] Data in data Frame file is double precision (1) or not (0).  Default: 0.
\item[add2channels] Keep 0, unless you want to read a signal from file.  Default: 0.

\item[noiseGPSstart] GPS time to start reading data to generate a noise PSD,  \textit{e.g.} 894377200.
\item[noisechannel] Name of the data channel in the Frame file, \textit{e.g.} \texttt{H1:STRAIN}.
\item[noisefilepath] Directory of the path in \texttt{SPINspiral.input.system} where the noise files sit.  Use ``.'' for no subdirectory.  Default: ``.''.
\item[noisefileprefix] Prefix of the Frame data file name, \textit{e.g.} \texttt{H-H1\_NINJA\_NOISE}.
\item[noisefilesuffix] Suffix of the Frame data file name, \textit{e.g.} \texttt{-1024.gwf}.
\item[noisefilesize] `Size' (in seconds) of the data in each noise Frame file,  \textit{e.g.} 1024.
\item[noisefileoffset]  Modulo between the start time and length of a Frame file.  If the Frame file name ends in: \texttt{-839366009-128.gwf}, \texttt{fileoffset} = \texttt{mod(839366009,128)},  \textit{e.g.} 743.
\item[noisedoubleprecision] Data in data Frame file is double precision (1) or nor (0).  Default: 0.
\end{description}






\pagebreak
\subsection{SPINspiral.input.injection}
The software injection input file


\subsubsection{General}
\begin{description}
\item[injectSignal] Inject a signal into the data (1) or not (0).
\item[injectionWaveform] Waveform version used for the software injection:  1 for 1.5PN 12-parameter Apostolatos, 2 for 3.5PN 12-parameter LAL, 3 for 3.5PN 15-parameter LAL.
\item[injectionPNorder] Post-Newtonian order at which the injection signal should be generated, \textit{e.g.} 1.5, 2.0, 3.5.
\item[injectionSNR] If $>0$: scale the distance such that the injection network SNR becomes injectionSNR.
\item[injRanSeed] Random number seed for random injection parameters. Don't change between serial chains of the same run!  Default: 12345.
\end{description}


\subsubsection{Table: Parameters}


Following the ``General'' section is a table with nine columns containing the following data:

\begin{description}
\item[Number] Currently just to guide the eye in this file.
\item[ID] a unique number for each parameter.  See Section~\ref{sec:parameter_catalogue} for a list of current IDs.
\item[InjectionValue] injection value for the parameter, can be overwritten depending on \texttt{ranInjPar}.
\item[RanInjPar] randomise the injection value:
  \begin{description}
  \item[0] no; inject InjectionValue.
  \item[1] yes; inject random value from a Gaussian distribution with centre InjectionValue and width Sigma (min BoundLow, max BoundUp).
  \item[2] yes; inject random value from range determined by BoundLow-BoundUp (make sure your MCMC prior matches this!).
  \end{description}
\item[Sigma] width of the Gaussian distribution to draw from for \texttt{RanInjPar=1}, \textit{e.g.} 0.1.

  
\item[BoundType] type of boundaries to use for the ranges of the injection parameters:
  \begin{description}
  \item[1] general range, BoundLow-BoundUp
  \item[2] general range, InjectionValue + BoundLow - InjectionValue + BoundUp; BoundLow must be $\leq 0$, BoundUp must be $\geq 0$.
  \item[3] general range, InjectionValue * BoundLow - InjectionValue * BoundUp; BoundLow must be $\leq 1$, BoundUp must be $\geq 1$.
  \end{description}
\item[BoundLow, Up]: used to determine upper or lower bound for BoundType = 1,2.
\item[Description] of the parameter, to increase readability of the file.
\end{description}                


   






\pagebreak

\subsection{SPINspiral.input.parameters}


\subsubsection{General}
\begin{description}                
\item[mcmcWaveform] Waveform version used as MCMC template:  1 for 1.5PN 12-parameter Apostolatos, 2 for 3.5PN 12-parameter LAL, 3 for 3.5PN 15-parameter LAL, 4 for non-spinning LAL.
\item[mcmcPNorder] Post-Newtonian order at which the MCMC template should be used, \textit{e.g.} 1.5, 2.0, 3.5.
\item[priorSet] Set of priors to use, currently: 1.
\item[offsetMCMC] Start the MCMC with offset initial parameters: 0-no: use injection parameters, overrules Start below; 1-yes: use Start below.  Default: 1.
\item[offsetX] Start the MCMC from a Gaussian distribution with a width of (offsetX times Sigma), if Start==2,4 below.  Default: 10.0.
\end{description}                





\subsubsection{Parameters}
\begin{description}                
\item[Number] Currently just to guide the eye in this file.
\item[ID] a unique number for a parameter.  See Section~\ref{sec:parameter_catalogue} for a list of current IDs.
\item[BestValue] best value for the parameter, \textit{e.g.} from the trigger;  use to start from or close to.
\item[Fix] fix an MCMC parameter (i.e., keep it constant throughout the MCMC run):
  \begin{description}                
  \item[0] no.
  \item[1] yes; fix to the starting value determined by \texttt{Start}
  \end{description}                
\item[Start] where to start the Markov chains:
  \begin{description}                
  \item[1] start at best value.
  \item[2] start near best value (Gaussian distribution with width sigma).
  \item[3] start at injection value.
  \item[4] start near injection value (Gaussian distribution with width sigma).
  \item[5] start randomly from range determined by BoundLow-BoundUp (see prior type).
  \end{description}                
\item[Sigma] width of the Gaussian distribution to start from for start=2,4; also used for diagonal of first correlation matrix.
\end{description}                

\subsubsection{Priors}
\begin{description}                
\item[Type] Type of prior range:
  \begin{description}                
  \item[11] general range, BoundLow-BoundUp.
  \item[12] general range, best value+BoundLow - best value+BoundUp; BoundLow must be $\leq 0$, BoundUp must be $\geq 0$.
  \item[13] general range, best value*BoundLow - best value*BoundUp; BoundLow must be $\leq 1$, BoundUp must be $\geq 1$. 
  \item[14] general range, injection value+BoundLow - injection value+BoundUp; BoundLow must be $\leq 0$, BoundUp must be $\geq 0$.
  \item[15] general range, injection value*BoundLow - injection value*BoundUp; BoundLow must be $\leq 1$, BoundUp must be $\geq 1$. 
  \item[21] periodic boundaries $0-2\pi$; BoundLow, BoundUp will be ignored.
  \item[22] periodic boundaries $0-\pi$; BoundLow, BoundUp will be ignored.
  \end{description}                
\item[BoundLow/Up] used to determine upper or lower bound, depending on prior type.  Ignored for periodic boundaries.
\item[Description] symbol/name of the parameter, to increase readability of the file.
\end{description}                



\pagebreak
\subsection{SPINspiral.input.system}
A file with system-dependent parameters (well, currently one).

\begin{description}
\item[datadir] Data directory (actual data files may be in a subdirectory of this, see \texttt{SPINspiral.input.data}), \textit{e.g.} \texttt{/home/user/MCMC/data}
\end{description}







\pagebreak
\section{Output files}

\subsection{SPINspiral.output.abcdef.xy}





\pagebreak
\section{Parameter catalogue}
\label{sec:parameter_catalogue}
Goal: assign a unique identifier to each possible variable that has been used in \textsc{SPINspiral} so far.
The identifier is stored in the variable \texttt{parID[]} and \texttt{injID[]} in the \texttt{runPar} and 
\texttt{mcmcvariables} structs.  The reverse identification is stored in \texttt{parRevID[]} and 
\texttt{injRevID[]}.  Thus, if the first parameter is $t_\mathrm{c}$, then \texttt{parID[0]}=11 and
\texttt{parRevID[11]}=0.  Whether a parameter is used (1) or not (0) for the software injection or MCMC 
is stored in \texttt{injParUse[]} and \texttt{mcmcParUse[]}.  The arrays \texttt{parAbrev[]} and 
\texttt{parAbrv[]} contain short and ultrashort parameter names respectively. The variable \texttt{parDef[i]} 
signifies whether the parameter with ID $i$ is defined (1) or not (0).

\subsection{Time}
\begin{tabular}{lllll}
  \textbf{Number} & \textbf{Variable} & \textbf{Description} & \textbf{Unit} & \textbf{Range} \\
  \textbf{11} & $t_\mathrm{c}$        & GPS time of coalescence & s & $[0,\infty[$ \\
  \textbf{12} & $t_\mathrm{40\,Hz}$   & GPS time at 40\,Hz & s & $[0,\infty[$ \\
\end{tabular}

\subsection{Distance}
\begin{tabular}{lllll}
  \textbf{Number} & \textbf{Variable} & \textbf{Description} & \textbf{Unit} & \textbf{Range} \\
  \textbf{21} & $d_\mathrm{L}^3$ & Luminosity distance & Mpc$^3$ & $]0,\infty[$ \\
  \textbf{22} & $^\mathrm{e}\log(d_\mathrm{L}/\mathrm{Mpc})$ & Luminosity distance & --- & $]-\infty,\infty[$ \\
\end{tabular}

\subsection{Sky position}
\begin{tabular}{lllll}
  \textbf{Number} & \textbf{Variable} & \textbf{Description} & \textbf{Unit} & \textbf{Range} \\
  \textbf{31} & R.A. ($\alpha$)  & Right ascension & rad & $[0,2\pi[$ \\
  \textbf{32} & $\sin(\mathrm{Dec})$ ($\sin(\delta)$)  & Declination & --- & $[-1,1]$ \\
\end{tabular}


\subsection{Phase}
\begin{tabular}{lllll}
  \textbf{Number} & \textbf{Variable} & \textbf{Description} & \textbf{Unit} & \textbf{Range} \\
  \textbf{41} & $\varphi_\mathrm{orb,1}$  & Orbital phase & rad & $[0,2\pi[$ \\
\end{tabular}

\subsection{Orientation}
\begin{tabular}{lllll}
  \textbf{Number} & \textbf{Variable} & \textbf{Description} & \textbf{Unit} & \textbf{Range} \\
  \textbf{51} & $\cos(\iota)$  & Inclination & --- & $[-1,1]$ \\
  \textbf{52} & $\psi$  & Polarisation angle & rad & $[0,\pi[$ \\
  \textbf{53} & $\sin(\theta_{J_0})$ & `Declination of $J_0$' & --- & $[-1,1]$ \\
  \textbf{54} & $\varphi_{J_0}$  & `R.A. of $J_0$' & rad & $[0,2\pi[$ \\
\end{tabular}

\subsection{Mass}
\begin{tabular}{lllll}
  \textbf{Number} & \textbf{Variable} & \textbf{Description} & \textbf{Unit} & \textbf{Range} \\
  \textbf{61} & $\cal{M}$  & Chirp mass & $M_\odot$ & $[0,\infty[$ \\
  \textbf{62} & $\eta$  & Symmetric mass ratio & --- & $[0,0.25]$ \\
  \textbf{63} & $M_1$  & Mass 1 & $M_\odot$ & $[0,\infty[$ \\
  \textbf{64} & $M_2$  & Mass 2 & $M_\odot$ &  $[0,\infty[$ \\
\end{tabular}

\subsection{Spin 1}
\begin{tabular}{lllll}
  \textbf{Number} & \textbf{Variable} & \textbf{Description} & \textbf{Unit} & \textbf{Range} \\
  \textbf{71} & $a_\mathrm{spin,1}$  & Spin magnitude 1 & --- & $[0,1]$ \\
  \textbf{72} & $\cos(\theta_\mathrm{spin,1})$  & Spin tilt 1 & --- & $[-1,1]$ \\
  \textbf{73} & $\varphi_\mathrm{spin,1}$  ($\alpha_\mathrm{c}$)  & Spin phase 1 & rad & $[0,2\pi[$ \\
      & & & & \\
  \textbf{75} & $S_\mathrm{1,x}$  & Spin 1 magnitude, x-component & --- & $[0,1]$ \\
  \textbf{76} & $S_\mathrm{1,y}$  & Spin 1 magnitude, y-component & --- & $[0,1]$ \\
  \textbf{77} & $S_\mathrm{1,z}$  & Spin 1 magnitude, z-component & --- & $[0,1]$ \\
\end{tabular}

\subsection{Spin 2}
\begin{tabular}{lllll}
  \textbf{Number} & \textbf{Variable} & \textbf{Description} & \textbf{Unit} & \textbf{Range} \\
  \textbf{81} & $a_\mathrm{spin,2}$  & Spin magnitude 2 & --- & $[0,1]$ \\
  \textbf{82} & $\cos(\theta_\mathrm{spin,2})$  & Spin tilt 2 & --- & $[-1,1]$ \\
  \textbf{83} & $\varphi_\mathrm{spin,2}$  & Spin phase 2 & rad & $[0,2\pi[$ \\
      & & & & \\
  \textbf{85} & $S_\mathrm{2,x}$  & Spin 2 magnitude, x-component & --- & $[0,1]$ \\
  \textbf{86} & $S_\mathrm{2,y}$  & Spin 2 magnitude, y-component & --- & $[0,1]$ \\
  \textbf{87} & $S_\mathrm{2,z}$  & Spin 2 magnitude, z-component & --- & $[0,1]$ \\
\end{tabular}

\subsection{Merger, ringdown}
\begin{tabular}{lllll}
  \textbf{Number} & \textbf{Variable} & \textbf{Description} & \textbf{Unit} & \textbf{Range} \\
  \textbf{91} &   & & & $[,]$ \\
\end{tabular}








\pagebreak
\section{Reading Frame files}
\label{sec:frame}

Frame files are designed to store detector data and are also used to store synthetic noise.
They can be read using \texttt{libframe} \cite{FrameLibrary}, see \textit{e.g.}
\texttt{/export/apps/lsc/libframe/} on Fugu.  

\textsc{SPINspiral} can read Frame files.  The parameters in \texttt{SPINspiral.input.data}
(Sect.\,\ref{sec:input.data.ifo}) determine how this is done.

The name of a Frame file typically looks like \texttt{H-H1\_SOME\_TEXT-123456789-123.gwf}
(using a file with detector data from H1 as an example).  The first part (\texttt{H-H1\_SOME\_TEXT-})
goes into the parameter \texttt{ch1fileprefix}, while the last part (\texttt{-123.gwf}) 
is stored in \texttt{ch1filesuffix}.

The example number \texttt{123456789} indicates the starting GPS time (in seconds) of the 
data in the file, while \texttt{123} is the length of the data strech in the file, again 
in seconds (and again an example). In order for the code to know at which GPS time Frame
files start and stop, we need to feed the modulus of the two numbers (\texttt{mod(123456789,123) = 90}
here) to the parameter \texttt{ch1fileoffset}.  At the moment, we cannot read subsequent files
of different length (using \texttt{FrCopy}, which can copy multiple files into one, is a 
quick-and-dirty solution. The ability to read cache files would be a better one).

One of the more obscure parameters is the channel name \texttt{ch1name}.  It can be found 
using the program \texttt{FrDump}, which comes with \texttt{libframe}: \\
\texttt{FrDump -i H-H1\_SOME\_TEXT-123456789-123.gwf |grep ProcData}\\
The channel name is probably going to look like \texttt{H1:STRAIN} (for the case of
the H1 detector).



\pagebreak
\section*{References}
\bibliographystyle{iopart-num}  %Nicer...
%\bibliographystyle{unsrt}  %Includes titles as well, messes up my et al solutions, forgets URLs.  Needed for online PDF generation at IOP
\bibliography{SPINspiral}


\end{document}




